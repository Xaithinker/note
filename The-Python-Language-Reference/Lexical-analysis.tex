\documentclass[10pt,UTF8]{ctexart}
\usepackage{amsmath}
\usepackage{graphicx}
\usepackage{hyperref}
\hypersetup{ colorlinks=true, linkcolor=blue, filecolor=gray, urlcolor=blue, citecolor=blue, }
\title{词法分析器}
\author{ZhangXu}
\begin{document}
\maketitle
词法分析器产生$tokens$,之后输入$parser$。此章节描述词法分析器如何将文件内容拆分成tokens。\\
\indent Python以Unicode编号形式读取程序文件,定义编码声明对文件编码,默认的是UTF-8(PEP 3120)否则SyntaxError异常产生。
\section{行结构}
Python程序被划分为\textbf{逻辑行}
\subsubsection{逻辑行}
逻辑行的结尾以token $NEWLINE$表示。逻辑行可由物理行通过显示或隐式的连接线规则构造。
\subsubsection{物理行}
物理行是由行尾序列终止的字符序列。Unix使用ASCII LF,Windows使用ASCII CR LF,Macintosh 使用 ASCII CR。\textbf{无关平台,所有这些可以平等使用}。输入的结尾充当物理行的隐式终止符。\\
\indent 当嵌入Python时,源代码字符应该使用标准C约定传递给Python API($\backslash $n字符表示ASCII LF,行终止符)。
\subsection{注释}
注释使用$\#$号来开始,物理行结尾。注释并非tokens
\subsection{编码声明}
若注释再Python文件的第一或第二行中,且符合正则匹配规则\[coding[=:]\backslash s*(-\backslash w.]+)\]则解析为编码声明;推荐使用以下格式\[\# -*- coding:<encoding-name> -*-\] \[\# vim:fileencoding=<encoding-name>\]
默认的是UTF-8。(若UTF-8字节串以b'$\backslash$xef$\backslash$xbb$\backslash$xbf'开头,微软的笔记本支持此)
\subsubsection{显式行连接}
通过$\backslash(backslash)$物理行可连接成一个逻辑行
\subsubsection{隐式行连接}
圆括号,方括号或大括号看看可以在使用反斜杠的情况下分割多个物理行。
\subsubsection{空行}
仅包含空格、制表符、换页符和可能的注释信息的逻辑行将被忽略(即不产生NEWLINE token)
\subsubsection{缩进}
逻辑行开头的空格或制表符用于计算缩进级别,确定语句的分组。缩进不能使用斜线分割多个物理行。\\
\indent 同时使用tabs和空格spaces会触发TabError异常。\\
\indent 连续行的缩进级别用来产生INDENT和DEDENT token,其使用堆栈数据结构。如下描述:\\
\indent 读入文件第一行之前,堆栈push一个永远不会被pop的0。push至堆栈的数字从栈底到栈顶严格增加。每个逻辑行的开头,将行的缩进级别与栈定比较。如果相等,保持原状。若缩进级别大于栈定数值,其将被push至栈顶且INDENT token产生。反过来若小于栈顶值,那么此缩进数值必须是堆栈中的一个数字;所有比缩进数值大的数将被pop直至相等,每个被pop的数值会产生一个DEDENT token。在文件的末尾,为留在堆栈中每个大于0的数值产生一个DEDENT token。
\subsubsection{tokens间的空格}
空格,tab,换页符均可用于拆分token成为两个token。
\section{其他tokens}
除了NEWLINE INDENT DEDENT,其他tokens如下:\[identifiers, keywords, literals, operators, and delimiters\](标识符,关键字,字面值,操作符,分隔符)。空格不是tokens,其可分隔tokens。在存在歧义的地方,token包括从左至右可组成的合法token的最长字符串。
\section{标识符和关键字}
标识符也被成为\textbf{名称}\\
\indent python3引入了ASCII范围之外的其他字符。对于这些字符,分类使用$unicodedata$模块中包含的Unicode字符数据库的版本。\\
\indent 标识符长度不受限。\\
\indent 所有的标识符在解析时被转换成\textbf{正规形式NFKC};通过\textbf{NFKC正规形式}比较标识符。\href{https://zh.wikipedia.org/wiki/Unicode%E7%AD%89%E5%83%B9%E6%80%A7}{Unicode等价性}
\subsubsection{关键字}
下列标识符被用作\textbf{保留字}或\textbf{关键字}。\\
\begin{center}
\begin{tabular}{ccccc}
\hline 
False & await & else & import & pass \\ 
\hline 
None & break & except & in & raise \\ 
\hline 
True & class & finally & is & return \\ 
\hline 
and & continue & for & lambda & try \\ 
\hline 
as & def & from & nonlocal & while \\ 
\hline 
assert & del & global & not & with \\ 
\hline 
async & elif & if & or & yield \\ 
\hline 
\end{tabular}
\end{center}
\subsubsection{保留的类标识符}
\paragraph{\_*}不能被\textbf{$from module import *$}导入。交互式解释器中使用特殊标识符$\_$来存储上次计算的结果;它存储在builtins模块中。当不处于交互模式时,$\_$没有特殊含义,也没有定义。
\paragraph{\_\_*\_\_}系统定义名称。由解释器或标准库定义名称。
\paragraph{\_\_*}类私有名称。
\section{字面值}
字面值指一些内置类型的常量值。
\subsubsection{字符串和字节串字面值}
字符串文字由一下词法定义描述:
\indent 无论字符串或字节串的前缀和剩下的字面值均不允许有空格。默认是UTF-8编码的字符串。\\
\indent 两种类型的文字均可用单引号或双引号括起来。也可以包含在三个单引号或双引号的组中(称其为$triple-quoted strings$)。$\backslash$用来转义特殊意思的字符。\\
\indent \textbf{字节串总是以'b'或'B'为前缀};其产生一个$bytes$类型的实例而不是$str$类型。它们包含ASCII字符,数字值大于128的字节必须用转义表示。\\
\indent 字符串或字节串都可以选择以字母'r'或'R'作前缀;称之为$\textbf{raw strings}$且视反斜杠为字面字符。(例如,'$\backslash$U'和'$\backslash$u'在raw strings中不再特殊对待。)\\
\indent 带有前缀'f'或'F'的字符串称\textbf{formatted string literal}。'f'可以和'r'组合,但是不能和'b'或'u'组合,因此raw formatted strings是可行的,但是格式化的字节串则不可行!\\
\indent 在triple-quoted文字中,不必转义换行符和引号。除非'r'或'R'前缀存在,否则转义序列将根据与标准C使用的类似规则解释。如下:
\begin{itemize}
\item $\backslash$newline 反斜杠和换行符被忽略
\item $\backslash$ $\backslash$ 代表反斜杠
\item $\backslash$' 代表单引号
\item $\backslash$" 双引号
\item $\backslash$a ASCII Bell(BEL)
\item $\backslash$b ASCII Backspace (BS)
\item $\backslash$f ASCII Formfeed (FF)
\item $\backslash$n ASCII Linefeed (LF)
\item $\backslash$r ASCII Carriage Return (CR)
\item $\backslash$t ASCII Horizontal Tab (TAB)
\item $\backslash$v ASCII Vertical Tab (VT)
\item $\backslash$ooo Character with octal value ooo
\item $\backslash$xhh Character with hex value hh
\end{itemize}
仅在字符串中识别的转义序列为:
\begin{itemize}
\item $\backslash$N{name} 
\item $\backslash$uxxx 16位十六进制xxxx的字符
\item $\backslash$Uxxx 32位十六进制的字符
\end{itemize}

\subsubsection{字符串连接}
使用''或""相邻的字符串会连接成一个整体。此特性定义在词法等级,在\textbf{编译时}实现;而"+"号在\textbf{运行时}实现。
\subsubsection{格式化字符串}
'f'或'F'开头的字符串表示格式化字符串。其可能包含可替代字段,由大括号分隔。其他字符串总是具有常量值,而\textbf{格式化字符串是在运行时计算的表达式}。
花括号外的部分按字面值处理,任何"双大括号"被替换成相应的单花括号。
替换部分由'{'开头,之后是\textbf{一个表达式},表达式后可能有一个'!'号引入的\textbf{转换字段(s,r,a)}。
之后可能跟一个由':'号引入的格式说明符(format specifier),最后由'}'闭合。\\
\indent 其中的表达式被视为由\textbf{括号}括起来的Python常规表达式。不允许使用空表达式,且Lambda表达式必须显式括起来。替换表达式看可以包含换行符,但是不能包含注释。表达式从左至右计算。\\
\indent 若指定了转换,格式化之前需计算表达式结果。'!s'使结果调用str(),'!r'调用repr(),'!a'调用ascii()。\\
\indent 结果出来后使用format()协议。格式说明符(例\%d,\# 0x)传递给表达式或转换结果的\_\_format\_\_()方法。然后得出最终结果。\\
\indent 顶级格式说明符包含嵌套替换区域。嵌套区域也可能包含它们自己的转换字段和格式说明符,但不能包含更深层次的替换字段。\href{https://docs.python.org/3/library/string.html#formatspec}{The format specifier mini-language}与字符串的.format()使用方法相同。\\
\indent 格式化字符串可以相连,但替换区域不可跨文字分割。\\
\indent 格式表达式中不允许使用反斜杠。需要包含反斜杠转义的值,创建一个临时变量。\\
\indent 格式化字符串文字不能用作文档字符串,即使它们仅是字符串不包含表达式。
\subsubsection{数字字面值}
三种:整数,浮点数和虚数。没有复数的直接定义,但可以使用一个实数和虚数。\\
\indent \textbf{注意数字字面值不包括符号,因此-1实际上是一个表达式,由'-'和1字面值组成。}
\subsubsection{整数字面值}
定义上整数没有限制大小。\textbf{确定字面值的数值时会忽略下划线}。(仅为分组提高数字可读性)数字之间可出现一个下划线,且在如0x之类的说明符之后。\\
\indent 非零十进制数的前导零不被允许使用。
\indent 整数包括十进制整数(100\_985\_211),二进制整数(0b 0B 0b\_ 0B\_001\_001),八进制整数(0-7)如上,十六进制整数(0-9 a-f)
\subsubsection{浮点数}
举例:3.14 10. .001 1e100 3.14e-10 0e0 3.14\_13\_33 3.1415e13\_31
\subsubsection{虚数}
形如(3+4j)创造一个实部非零的复数。

\section{操作符}
\begin{center}
\begin{tabular}{cccccc}
\hline 
+ & - & * & ** & / & // \\ 
\hline 
\% & @ & $\gg$ & $\ll$ & \& & | \\ 
\hline 
$\wedge$ & $\sim$ & $<$ & $>$ & $\le$ & $\ge$ \\
\hline 
== & !=  \\ 
\hline 
\end{tabular} 
\end{center}
\section{分隔符}
\begin{center}
\begin{tabular}{ccccccc}
\hline 
( & ) & [ & ] & \{ & \} &  \\ 
\hline 
, & : & . & ; & @ & = & $\rightarrow$ \\ 
\hline 
$+=$ & $-=$ & $*=$ & $/=$ & $//=$ & $\%=$ & $@=$ \\ 
\hline 
$\%=$ & $|=$ & $\wedge=$ & $\gg=$ & $\ll=$ & $**=$ &  \\ 
\hline 
\end{tabular}
\end{center}
\end{document}