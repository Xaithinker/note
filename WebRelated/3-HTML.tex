\documentclass[10pt,UTF8]{ctexart}
\usepackage{amsmath}
\usepackage{graphicx}
\usepackage{hyperref}
\hypersetup{ colorlinks=true, linkcolor=blue, filecolor=gray, urlcolor=blue, citecolor=blue, }
\title{More on HTML}
\author{ZhangXu}
\begin{document}
\maketitle
\newcounter{counter}

\section{XHTML}
\begin{itemize}
\item XHTML 指的是可扩展超文本标记语言
\item XHTML 与 HTML 4.01 几乎是相同的
\item XHTML 是更严格更纯净的 HTML 版本
\item XHTML 是以 XML 应用的方式定义的 HTML
\item HTML 是 2001 年 1 月发布的 W3C 推荐标准
\item XHTML 得到所有主流浏览器的支持
\end{itemize}
\subsection{为什么使用XHTML}
因特网上的很多页面包含了“糟糕”的 HTML。\\
XML 是一种必须正确标记且格式良好的标记语言。\\
通过结合 XML 和 HTML 的长处,开发出了 XHTML。XHTML 是作为 XML 被重新设计的 HTML。
\subsection{区别}
\paragraph{文档结构}
\begin{itemize}
\item XHTML DOCTYPE 是强制性的
\item $<html>$ 中的 XML namespace 属性是强制性的
\item $<html>$、$<head>$、$<title>$ 以及 $<body>$ 也是强制性的
\end{itemize}
\paragraph{元素语法}
\begin{itemize}
\item XHTML 元素必须正确嵌套
\item XHTML 元素必须始终关闭
\item XHTML 元素必须小写
\item XHTML 文档必须有一个根元素
\end{itemize}

\paragraph{属性语法}
\begin{itemize}
\item XHTML 属性必须使用小写
\item HTML 属性值必须用引号包围
\item XHTML 属性最小化也是禁止的
\end{itemize}
\end{document}