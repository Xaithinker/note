\documentclass[10pt,UTF8]{ctexart}
\usepackage{amsmath}
\usepackage{graphicx}
\usepackage{hyperref}
\hypersetup{ colorlinks=true, linkcolor=blue, filecolor=gray, urlcolor=blue, citecolor=blue, }
\title{HTML Related}
\author{ZhangXu}
\begin{document}
\maketitle
\newcounter{counter}

\section{前言}上承接Python基本的获取数据方法。关于数据的解析提取工作,可使用re,bs4,xpath等,暂时不深究。\\
\indent 接下来便是对动态网页的提取,觉得有必要事先了解一下HTML相关的内容。
\part{HTML/CSS}
\href{http://www.w3school.com.cn/html/index.asp}{w3school的教程} \\
\begin{itemize}
\item HTML 指的是超文本标记语言 (Hyper Text Markup Language)
\item HTML 不是一种编程语言,而是一种标记语言 (markup language)
\item 标记语言是一套标记标签 (markup tag)
\item HTML 使用标记标签来描述网页
\end{itemize}
\section{基础}
\paragraph{HTML标题} HTML标题(Heading)是通过 $<h1>$ - $<h6>$ 等标签进行定义的。
\paragraph{段落} $<p>$标签定义
\paragraph{链接} $<a href="">$ content $</a>$
\paragraph{图像} $<img src="" width="" height="" />$

\section{元素}
HTML元素指的是从开始标签(start tag)到结束标签(end tag)的所有代码。其定义HTML文档。
\subsection{元素语法}
\begin{itemize}
\item \textbf{开始标签}起始
\item \textbf{结束标签}终止
\item \textbf{元素的内容}是开始标签与结束标签之间的内容
\item 某些 HTML 元素具有\textbf{空内容}(empty content)
\item 空元素在\textbf{开始标签中进行关闭}(以开始标签的结束而结束)
\item 大多数 HTML 元素可拥有\textbf{属性}
\end{itemize}

\subsection{空元素}
没有内容的 HTML 元素被称为空元素。空元素是在开始标签中关闭的。\\
$<br>$ 就是没有关闭标签的空元素($<br>$ 标签定义换行)。\\
在 XHTML、XML 以及未来版本的 HTML 中,所有元素都必须被关闭。\\

在开始标签中添加斜杠,比如 $<br />$,是关闭空元素的正确方法,HTML、XHTML 和 XML 都接受这种方式。\\

即使 $<br>$ 在所有浏览器中都是有效的,但使用 $<br />$ 其实是更长远的保障。
\subsection{TIPs}
HTML 标签对大小写不敏感:$<P>$ 等同于 $<p>$。许多网站都使用大写的 HTML 标签。\\
万维网联盟(W3C)在 HTML 4 中推荐使用小写,而在未来 (X)HTML 版本中强制使用小写。


\section{HTML属性}
\noindent \textbf{属性为 HTML 元素提供附加信息。}
\subsection{属性}
属性总是以名称/值对的形式出现,比如:name="value"。\\
属性总是在 HTML 元素的开始标签中规定。

\subsection{TIPs}
属性和属性值对大小写不敏感。万维网联盟在其 HTML 4 推荐标准中推荐小写的属性/属性值。而新版本的 (X)HTML 要求使用小写属性。\\
\textbf{始终为属性值加引号}

\subsection{属性参考手册}
\noindent \href{http://www.w3school.com.cn/tags/index.asp}{完整的 HTML 参考手册}
\begin{itemize}
\item class classname 规定元素的类名(classname) 
\item id id 规定元素的唯一 id
\item style style\_definition 规定元素的行内样式(inline style)
\item title text 规定元素的额外信息(可在工具提示中显示
\end{itemize}
\section{标题}
\noindent 标题(Heading)是通过 $<h1>$ - $<h6>$ 等标签进行定义的。\\
搜索引擎使用标题为您的网页的结构和内容编制索引。\\
$<hr />$ 标签在 HTML 页面中创建水平线。hr 元素可用于分隔内容。\\
注释: $<!-- This is a comment -->$

\section{段落}
$<p>$标签定义。
\subsection{输出}
对于 HTML,您无法通过在 HTML 代码中添加额外的空格或换行来改变输出的效果。\\
当显示页面时,浏览器会移除源代码中多余的空格和空行。所有连续的空格或空行都会被算作一个空格。需要注意的是,HTML 代码中的所有连续的空行(换行)也被显示为一个空格。

\section{样式}
style 属性用于改变 HTML 元素的样式。
\subsection{style属性}
\textbf{提供了一种改变所有 HTML 元素的样式的通用方法}\\
通过 HTML 样式,能够通过使用 style 属性直接将样式添加到 HTML 元素,或者间接地在独立的样式表中(CSS 文件)进行定义。

\section{文本格式化}
\textbf{HTML 可定义很多供格式化输出的元素,比如粗体和斜体字。}.

\section{引用Quotation}
$<q>$元素定义\textbf{短的引用}。浏览器通常会为 $<q>$ 元素包围引号 \\
$<blockquote>$定义被引用的节。浏览器通常会对 $<blockquote>$ 元素进行\textbf{缩进}处理。 \\
$<abbr>$定义缩写或首字母缩略语。\\
$<dfn>$ \\
$<address>$定义文档联系作者,通常以斜体显示。\\
$<cite>$定义著作的标题

\section{计算机代码元素}
\paragraph{代码格式} 通常,HTML 使用可变的字母尺寸,以及可变的字母间距。在显示计算机代码示例时,并不需要如此。$<kbd>$, $<samp>$, 以及 $<code>$ 元素全都支持固定的字母尺寸和间距。
\paragraph{键盘格式}  $<kbd>$ 元素定义键盘输入
\paragraph{样本格式} $<samp>$ 元素定义计算机输出示例
\paragraph{代码格式}  $<code>$ 元素定义编程代码示例( 元素不保留多余的空格和折行,如需解决该问题,您必须在 $<pre>$ 元素中包围代码)
\paragraph{变量格式化} $<var>$ 元素定义数学变量
\section{注释}
注释标签 $<!-- 与 -->$ 用于在 HTML 插入注释
\paragraph{条件注释} $<!--[if IE 8]>$.... some HTML here ....$<![endif]-->$

\section{CSS}
所有的格式化代码均可移出 HTML 文档,然后移入一个独立的样式表。
\subsection{使用样式}
\begin{itemize}
\item \textbf{外部样式} $<head>$
$<link rel="stylesheet" type="text/css" href="mystyle.css"> </head>$
\item \textbf{内部样式表}$<head>$
$<style type="text/css">$
body {background-color: red}
p {margin-left: 20px}
$</style></head>$
\item \textbf{内联样式} $<p style="color: red; margin-left: 20px">$
This is a paragraph
$</p>$
\end{itemize}
\href{http://www.w3school.com.cn/css/index.asp}{CSS教}

\section{链接}
\paragraph{语法} $<a href="url">$Link text$</a>$
\paragraph{target属性} 使用 Target 属性,你可以定义被链接的文档在何处显示。target="\_blank"会在新窗口打开。
\subsection{name属性}
对读者不可见。\\
当使用命名锚(named anchors)时,我们可以创建直接跳至该命名锚(比如页面中某个小节)的链接,这样使用者就无需不停地滚动页面来寻找他们需要的信息了。
\paragraph{命名锚的语法} $<a name="label">$锚(显示在页面上的文本)$</a>$(您可以使用 id 属性来替代 name 属性,命名锚同样有效。)\\
将 \# 符号和锚名称添加到 URL 的末端,就可以直接链接到 tips 这个命名锚了。

\section{图像}
\paragraph{语法}$<img src="url" />$
\paragraph{替换文本}Alt在浏览器无法载入图像时,替换文本属性告诉读者她们失去的信息

\section{表格}
\subsection{表格}
表格由 $<table>$ 标签来定义。每个表格均有若干行(由 $<tr>$ 标签定义),每行被分割为若干单元格(由 $<td>$ 标签定义)。字母 td 指表格数据(table data),即数据单元格的内容。数据单元格可以包含文本、图片、列表、段落、表单、水平线、表格等等。
\subsection{表格和边框属性}
border="1"
\paragraph{表头}$<th>$ 标签进行定义。大多数浏览器会把表头显示为粗体居中的文本:

\section{列表}
\paragraph{无序列表} 无序列表是一个项目的列表,此列项目使用粗体圆点(典型的小黑圆圈)进行标记。无序列表始于 $<ul>$ 标签。每个列表项始于 $<li>$。
\paragraph{有序列表}同样,有序列表也是一列项目,列表项目使用数字进行标记。有序列表始于 $<ol>$ 标签。每个列表项始于 $<li>$ 标签
\paragraph{定义列表}自定义列表不仅仅是一列项目,而是项目及其注释的组合。自定义列表以 $<dl>$ 标签开始。每个自定义列表项以 $<dt>$ 开始。每个自定义列表项的定义以 $<dd>$ 开始。

\section{div span}
可以通过 $<div>$ 和 $<span>$ 将 HTML 元素组合起来
\subsection{块元素}
\begin{itemize}
\item 大多数 HTML 元素被定义为块级元素或内联元素。
\item 块级元素在浏览器显示时,通常会以新行来开始(和结束)。
\end{itemize}
\subsection{内联元素}
内联元素在显示时通常不会以新行开始。
\subsection{$<div>$元素}
\begin{itemize}
\item $<div>$ 元素是块级元素,它是可用于组合其他 HTML 元素的容器
\item 如果与 CSS 一同使用,$<div>$ 元素可用于对大的内容块设置样式属性。
\item $<div>$ 元素的另一个常见的用途是文档布局。它取代了使用表格定义布局的老式方法。使用 $<table>$ 元素进行文档布局不是表格的正确用法。$<table>$ 元素的作用是显示表格化的数据。
\end{itemize}
\subsection{$<span>$元素}
\begin{itemize}
\item $<span>$ 元素是内联元素,可用作文本的容器
\item $<span>$ 元素也没有特定的含义
\item 当与 CSS 一同使用时,$<span>$ 元素可用于为部分文本设置样式属性。
\end{itemize}

\subsection{分组标签}
\paragraph{div}定义文档中的分区或节(division/section)。
\paragraph{span}定义 span,用来组合文档中的行内元素

\section{类}
\noindent 对 HTML 进行分类(设置类),使我们能够为元素的类定义 CSS 样式。\\
为相同的类设置相同的样式,或者为不同的类设置不同的样式。
\paragraph{分级块元素}$<div>$ 元素是块级元素。它能够用作其他 HTML 元素的容器。设置 $<div>$ 元素的类,使我们能够为相同的 $<div>$ 元素设置相同的类。

\section{布局}
\subsection{div元素布局HTML}
$<div>$ 元素常用作布局工具,因为能够轻松地通过 CSS 对其进行定位。
\begin{itemize}
\item header 定义文档或节的页眉
\item nav 定义导航链接的容器
\item section 定义文档中的节
\item article 定义独立的自包含文章
\item aside	定义内容之外的内容(比如侧栏)
\item footer 定义文档或节的页脚
\item details 定义额外的细节
\item summary 定义 details 元素的标题
\end{itemize}

\subsubsection{使用表格的HTML布局}
\paragraph{注释:} $<table>$ 元素不是作为布局工具而设计的。$<table>$ 元素的作用是显示表格化的数据。使用 $<table>$ 元素能够取得布局效果,因为能够通过 CSS 设置表格元素的样式

\section{响应式Web设计}
\subsection{概念}
\begin{itemize}
\item RWD 指的是响应式 Web 设计(Responsive Web Design)
\item RWD 能够以可变尺寸传递网页
\item RWD 对于平板和移动设备是必需的
\end{itemize}

\subsubsection{使用Bootstrap}
Bootstrap 是最流行的开发响应式 web 的 HTML, CSS, 和 JS 框架。
Bootstrap 帮助您开发在任何尺寸都外观出众的站点:显示器、笔记本电脑、平板电脑或手机

\section{框架}
通过使用框架,你可以在同一个浏览器窗口中显示不止一个页面。每份HTML文档称为一个框架,并且每个框架都独立于其他的框架。
\begin{itemize}
\item 开发人员必须同时跟踪更多的HTML文档
\item 很难打印整张页面
\item 框架结构标签($<frameset>$)定义如何将窗口分割为框架
\item 每个 frameset 定义了一系列行或列
\item rows/columns 的值规定了每行或每列占据屏幕的面积
\end{itemize}
\subsection{Frame标签}
Frame 标签定义了放置在每个框架中的 HTML 文档\\
假如一个框架有可见边框,用户可以拖动边框来改变它的大小。为了避免这种情况发生,可以在 $<frame>$ 标签中加入:noresize="noresize"。

\section{iframe}
\paragraph{语法}$<iframe src="URL"> </iframe>$
\subsection{iframe 高度和宽度}
height 和 width 属性用于规定 iframe 的高度和宽度。\\
属性值的默认单位是像素,但也可以用百分比来设定(比如 "80\%")
\subsection{删除边框}frameborder 属性规定是否显示 iframe 周围的边框。\\
设置属性值为 "0" 就可以移除边框

\section{背景}

应该使用层叠样式表(CSS)来定义 HTML 元素的布局和显示属性。

\section{脚本}
JavaScript 使 HTML 页面具有更强的动态和交互性
\paragraph{scrpt元素}
\begin{itemize}
\item $<script>$ 标签用于定义客户端脚本,比如 JavaScript
\item script 元素既可包含脚本语句,也可通过 src 属性指向外部脚本文件
\item 必需的 type 属性规定脚本的 MIME 类型
\item JavaScript 最常用于图片操作、表单验证以及内容动态更新
\item $<noscript>$ 标签提供无法使用脚本时的替代内容,比方在浏览器禁用脚本时,或浏览器不支持客户端脚本时。\\
noscript 元素可包含普通 HTML 页面的 body 元素中能够找到的所有元素。\\
只有在浏览器不支持脚本或者禁用脚本时,才会显示 noscript 元素中的内容
\end{itemize}

\subsection{对付老式浏览器}
如果浏览器压根没法识别 $<script>$ 标签,那么 $<script>$ 标签所包含的内容将以文本方式显示在页面上。为了避免这种情况发生,你应该将脚本隐藏在注释标签当中。那些老的浏览器(无法识别 $<script>$ 标签的浏览器)将忽略这些注释,所以不会将标签的内容显示到页面上。而那些新的浏览器将读懂这些脚本并执行它们,即使代码被嵌套在注释标签内。

\section{头部元素}
\paragraph{$<head>$元素} $<head>$ 元素是所有头部元素的容器。$<head>$ 内的元素可包含脚本,指示浏览器在何处可以找到样式表,提供元信息,等等。\\
以下标签都可以添加到 head 部分:$<title>$、$<base>$、$<link>$、$<meta>$、$<script>$ 以及 $<style>$
\paragraph{title}
\begin{itemize}
\item $<title>$ 标签定义文档的标题
\item title 元素在所有 HTML/XHTML 文档中都是必需的
\item 定义浏览器工具栏中的标题
\item 提供页面被添加到收藏夹时显示的标题
\item 显示在搜索引擎结果中的页面标题
\item
\item
\item
\item
\end{itemize}
\paragraph{$<base>$元素} 标签为页面上的所有链接规定默认地址或默认目标(target)
\paragraph{$<link>$元素} 标签定义文档与外部资源之间的关系(标签最常用于连接样式表)
\paragraph{$<style>$元素} 标签用于为 HTML 文档定义样式信息。
\paragraph{$<meta>$元素} 元数据(metadata)是关于数据的信息。标签提供关于 HTML 文档的元数据。元数据不会显示在页面上,但是对于机器是可读的。\\
典型的情况是,meta 元素被用于规定页面的描述、关键词、文档的作者、最后修改时间以及其他元数据。$<meta>$ 标签始终位于 head 元素中。元数据可用于浏览器(如何显示内容或重新加载页面),搜索引擎(关键词),或其他 web 服务
\subparagraph{针对搜索引擎关键词} 一些搜索引擎会利用 meta 元素的 name 和 content 属性来索引您的页面

\section{字符实体}
中的预留字符必须被替换为字符实体
\paragraph{实体}如果希望正确地显示预留字符,我们必须在 HTML 源代码中使用字符实体(character entities)\&entity\_name;\&\#entity\_number;
\paragraph{不间断空格}HTML 中的常用字符实体是不间断空格(\&nbsp;)浏览器总是会截短 HTML 页面中的空格。如果您在文本中写 10 个空格,在显示该页面之前,浏览器会删除它们中的 9 个。如需在页面中增加空格的数量,您需要使用 \&nbsp; 字符实体

\section{URL}
\begin{itemize}
\item scheme - 定义因特网服务的类型。最常见的类型是 http
\item host - 定义域主机(http 的默认主机是 www)
\item domain - 定义因特网域名,比如 w3school.com.cn
\item :port - 定义主机上的端口号(http 的默认端口号是 80)
\item path - 定义服务器上的路径(如果省略,则文档必须位于网站的根目录中)。
\item filename - 定义文档/资源的名称
\end{itemize}
(经试验,urlencode按照网页要求,一般是gbk编码或utf-8编码,然后每个非ascii码要添加百分号前缀。)

\section{URL字符编码}
URL 编码会将字符转换为可通过因特网传输的格式
\paragraph{URL编码}URL 只能使用 ASCII 字符集来通过因特网进行发送。由于 URL 常常会包含 ASCII 集合之外的字符,URL 必须转换为有效的 ASCII 格式。URL 编码使用 "\%" 其后跟随两位的十六进制数来替换非 ASCII 字符。URL 不能包含空格。URL 编码通常使用 + 来替换空格。

\section{Web Server}
如果希望向世界发布您的网站,那么您必须把它存放在 web 服务器上
\begin{itemize}
\item 硬件支出
\item 软件支出
\item 人工费
\item 使用因特网服务提供商(ISP)
\end{itemize}

\section{颜色}
颜色由一个十六进制符号来定义,这个符号由红色、绿色和蓝色的值组成(RGB)。\\
每种颜色的最小值是0(十六进制:\#00)。最大值是255(十六进制:\#FF)。

\section{$<!DOCTYPE>$}
$<!DOCTYPE>$ 不是 HTML 标签。它为浏览器提供一项信息(声明),即 HTML 是用什么版本编写的。
\paragraph{HTML5}$<!DOCTYPE html>$

\section{HTML表单}
HTML 表单用于搜集不同类型的用户输入
\subsection{$<form>$元素}
元素定义 HTML 表单。HTML 表单包含表单元素。
\paragraph{$<input>$元素}最重要的表单元素。元素有很多形态,根据不同的 type 属性
\begin{itemize}
\item text:定义常规文本输入
\item radio:定义单选按钮输入(选择多个选择之一)
\item submit:定义提交按钮(提交表单)
\end{itemize}
\subsection{Action属性}
action 属性定义在提交表单时执行的动作
\subsection{Method属性}
method 属性规定在提交表单时所用的 HTTP 方法(GET 或 POST)
\subsection{$<select>$元素}
元素定义下拉列表
\subsection{$<option>$元素}
列表通常会把首个选项显示为被选选项。您能够通过添加 selected 属性来定义预定义选项。
\subsection{$<textarea>$元素}
元素定义多行输入字段(文本域)
\subsection{$<button>$}
定义可点击的按钮
\subsection{HTML5表单元素}
\begin{itemize}
\item $<datalist>$ 为 $<input>$ 元素规定预定义选项列表。$<input>$ 元素的 list 属性必须引用 <datalist> 元素的 id 属性。
\item $<keygen>$
\item $<output>$
\end{itemize}
\section{HTML输入类型}
\begin{itemize}
\item text
\item password
\item submit 定义提交表单数据至表单处理程序的按钮。表单处理程序(form-handler)通常是包含处理输入数据的脚本的服务器页面。在表单的 action 属性中规定表单处理程序(form-handler)
\item radio
\item checkbox 定义复选框
\item button
\item number 包括\textbf{输入限制}
\item date  用于应该包含日期的输入字段
\item color
\item range
\item month
\item week
\item time 用户选择时间(无时区)
\item datetime 用户选择日期和时间(有时区)
\item datetime-local 用户选择日期和时间(无时区)
\item email
\item search
\item tel
\item url
\end{itemize}
\section{HTML Input属性}
\subsection{value属性}
规定输入字段的初始值
\subsection{readonly 属性}
规定输入字段为只读(不能修改)
\subsection{disabled 属性}
规定输入字段是禁用的。被禁用的元素不会被提交。
\subsection{size 属性}
属性规定输入字段的尺寸(以字符计)
\subsection{maxlength 属性}
属性规定输入字段允许的最大长度
\subsection{autocomplete属性}
属性规定表单或输入字段是否应该自动完成。autocomplete 属性适用于 $<form>$ 以及如下 $<input>$ 类型:text、search、url、tel、email、password、datepickers、range 以及 color
\subsection{novalidate 属性}
novalidate 属性属于 <form> 属性。如果设置,则 novalidate 规定在提交表单时不对表单数据进行验证
\subsection{autofocus 属性}
autofocus 属性是布尔属性。如果设置,则规定当页面加载时 <input> 元素应该自动获得焦点

\subsection{form 属性}
属性规定$<input>$元素所属的一个或多个表单。输入字段位于 HTML 表单之外(但仍属表单)

\subsection{formaction 属性}
formaction 属性规定当提交表单时处理该输入控件的文件的 URL。\\
覆盖 <form> 元素的 action 属性。\\
适用于 type="submit" 以及 type="image"
\subsection{formenctype 属性}
属性规定当把表单数据(form-data)提交至服务器时如何对其进行编码(仅针对 method="post" 的表单。覆盖 <form> 元素的 enctype 属性。适用于 type="submit" 以及 type="image"
\subsection{formmethod 属性}
\subsection{formnovalidate 属性}
\subsection{formtarget 属性}
\subsection{height和width属性}
height 和 width 属性规定 <input> 元素的高度和宽度。仅用于 <input type="image">
\subsection{list属性}
list 属性引用的 <datalist> 元素中包含了 <input> 元素的预定义选项。
\subsection{min和max属性}
规定 <input> 元素的最小值和最大值。适用于如需输入类型:number、range、date、datetime、datetime-local、month、time 以及 week。
\subsection{multiple属性}
如果设置,则规定允许用户在 <input> 元素中输入一个以上的值。multiple 属性适用于以下输入类型:email 和 file
\subsection{pattern属性}
规定用于检查 <input> 元素值的正则表达式。适用于以下输入类型:text、search、url、tel、email、and password
\subsection{placeholder 属性}
规定用以描述输入字段预期值的提示(样本值或有关格式的简短描述)。该提示会在用户输入值之前显示在输入字段中。placeholder 属性适用于以下输入类型:text、search、url、tel、email 以及 password
\subsection{required 属性}
required 属性是布尔属性。如果设置,则规定在提交表单之前必须填写输入字段
\subsection{step 属性}
规定 <input> 元素的合法数字间隔

\section{HTML5 Related}
\subsection{figure figcaption元素}
通过 HTML5,图片和标题能够被组合在 $<figure>$元素中。$<img>$ 元素定义图像,$<figcaption>$ 元素定义标题
\subsection{Canvas}
canvas 元素用于在网页上绘制图形
\paragraph{什么是Canvas?} canvas 元素使用 JavaScript 在网页上绘制图像。画布是一个矩形区域,您可以控制其每一像素。canvas 拥有多种绘制路径、矩形、圆形、字符以及添加图像的方法
\paragraph{Canvas创建}
$<canvas id="myCanvas" width="200" height="100"></canvas>$
\paragraph{通过JS来绘制}canvas 元素本身是没有绘图能力的。所有的绘制工作必须在 JavaScript 内部完成
\section{内联SVG}
HTML5支持内联SVG
\subsection{SVG}
\begin{itemize}
\item 可伸缩矢量图形 (Scalable Vector Graphics)
\item 定义用于网络的基于矢量的图形
\item 使用 XML 格式定义图形
\item 在放大或改变尺寸的情况下其图形质量不会有损失
\item 万维网联盟的标准
\item 优势:
\item SVG 图像可通过文本编辑器来创建和修改
\item SVG 图像可被搜索、索引、脚本化或压缩
\item 是可伸缩的
\item 可在任何的分辨率下被高质量地打印
\item 可在图像质量不下降的情况下被放大
\end{itemize}
\section{Canvas vs. SVG}
\subsection{SVG}
\begin{itemize}
\item SVG 是一种使用 XML 描述 2D 图形的语言
\item SVG 基于 XML,这意味着 SVG DOM 中的每个元素都是可用的。您可以为某个元素附加 JavaScript 事件处理器
\item 在 SVG 中,每个被绘制的图形均被视为对象。如果 SVG 对象的属性发生变化,那么浏览器能够自动重现图形
\end{itemize}
\subsection{Canvas}
\begin{itemize}
\item Canvas 通过 JavaScript 来绘制 2D 图形
\item Canvas 是逐像素进行渲染的
\item 在 canvas 中,一旦图形被绘制完成,它就不会继续得到浏览器的关注。如果其位置发生变化,那么整个场景也需要重新绘制,包括任何或许已被图形覆盖的对象
\end{itemize}
\subsection{比较}
\noindent Canvas
\begin{itemize}
\item 依赖分辨率
\item 不支持事件处理器
\item 弱的文本渲染能力
\item 能够以 .png 或 .jpg 格式保存结果图像
\item 最适合图像密集型的游戏,其中的许多对象会被频繁重绘
\end{itemize}
\noindent SVG
\begin{itemize}
\item 不依赖分辨率
\item 支持事件处理器
\item 最适合带有大型渲染区域的应用程序(比如谷歌地图)
\item 复杂度高会减慢渲染速度(任何过度使用 DOM 的应用都不快)
\item 不适合游戏应用
\end{itemize}
\section{多媒体}
Web 上的多媒体指的是音效、音乐、视频和动画。现代网络浏览器已支持很多多媒体格式\\
WAVE 是因特网上最受欢迎的无压缩声音格式,所有流行的浏览器都支持它。如果您需要未经压缩的声音(音乐或演讲),那么您应该使用 WAVE 格式 \\
MP3 是最新的压缩录制音乐格式。MP3 这个术语已经成为数字音乐的代名词。如果您的网址从事录制音乐,那么 MP3 是一个选项。

\end{document}