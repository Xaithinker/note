\documentclass[10pt,UTF8]{ctexart}
\usepackage{amsmath}
\usepackage{graphicx}
\usepackage{hyperref}
\usepackage{underscore}
\hypersetup{ colorlinks=true, linkcolor=blue, filecolor=gray, urlcolor=blue, citecolor=blue, }
\title{HTML5 Related}
\author{ZhangXu}
\begin{document}
\maketitle
\newcounter{counter}
\section{HTML5 拖放}
\subsection{Drag 和 Drop}
它指的是您抓取某物并拖入不同的位置\\
拖放是 HTML5 标准的组成部分:任何元素都是可拖放的

\section{HTML本地存储}
HTML 本地存储:优于 cookies
\subsection{什么是本地存储}
通过本地存储(Local Storage),web 应用程序能够在用户浏览器中对数据进行本地的存储。在 HTML5 之前,应用程序数据只能存储在 cookie 中,包括每个服务器请求。本地存储则更安全,并且可在不影响网站性能的前提下将大量数据存储于本地。与 cookie 不同,存储限制要大得多(至少5MB),并且信息不会被传输到服务器。本地存储经由起源地(origin)(经由域和协议)。所有页面,从起源地,能够存储和访问相同的数据。
\section{HTML本地存储对象}
HTML 本地存储提供了两个在客户端存储数据的对象:\\
window.localStorage - 存储没有截止日期的数据\\
window.sessionStorage - 针对一个 session 来存储数据(当关闭浏览器标签页时数据会丢失)

\section{hTML5应用程序缓存}
\subsection{什么是应用程序缓存}
HTML5 引入了应用程序缓存(Application Cache),这意味着可对 web 应用进行缓存,并可在没有因特网连接时进行访问。\\
应用程序缓存为应用带来三个优势:
\begin{itemize}
\item[1] 离线浏览 - 用户可在应用离线时使用它们
\item[2] 速度 - 已缓存资源加载得更快
\item[3] 减少服务器负载 - 浏览器将只从服务器下载更新过或更改过的资源
\end{itemize}

\subsection{Cache Manifest基础}
如需启用应用程序缓存,请在文档的 <html> 标签中包含 manifest 属性\\
每个指定了 manifest 的页面在用户对其访问时都会被缓存。如果未指定 manifest 属性,则页面不会被缓存(除非在 manifest 文件中直接指定了该页面)\\
manifest 文件的建议文件扩展名是:".appcache"\\
manifest 文件需要设置正确的 MIME-type,即 "text/cache-manifest"。必须在 web 服务器上进行配置

\section{HTML Web Workers}
Web worker 是运行在后台的 JavaScript,不会影响页面的性能\\
当在 HTML 页面中执行脚本时,页面是不可响应的,直到脚本已完成。Web worker 是运行在后台的 JavaScript,独立于其他脚本,不会影响页面的性能。您可以继续做任何愿意做的事情:点击、选取内容等等,而此时 web worker 运行在后台。
\section{Server-Sent事件-One Way Messaging}
\begin{itemize}
\item 创建一个新的 EventSource 对象,然后规定发送更新的页面的 URL(本例中是 "demo_sse.php")
\item 每当接收到一次更新,就会发生 onmessage 事件
\item 当 onmessage 事件发生时,把已接收的数据推入 id 为 "result" 的元素中
\end{itemize}
\end{document}