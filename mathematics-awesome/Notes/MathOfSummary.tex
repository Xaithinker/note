\documentclass[10pt,a4paper,UTF8]{ctexart}
\usepackage{graphics}
\usepackage{amsmath}
\usepackage{bm}
\title{Mathematica}
\author{ZhangXu}
\date{\today}

\begin{document}
\maketitle

\section{极限与连续}
\newcounter{theoremcounter}
\setcounter{theoremcounter}{1}
\subsection{数列极限定理}
\newtheorem{theorem1}{数列极限}[theoremcounter]
\begin{theorem1}
对于任意的$\varepsilon>0$(不论它多么小),总存在正整数$N$,使得当
\\$n>N$时,$|x_n-a|<\varepsilon$恒成立,则称数$a$是数列$\{x_n\}$收敛于$a$,记为\\
\[ \lim_{n\rightarrow\infty}{x_n=a} \hspace{1pt}\textit{或}\hspace{1pt} x_n\rightarrow a(n\rightarrow \infty). \]
\\%edited here.12.23.2018 0:40
% Edit on 12.24.2018
如果不存在这样的数$a$,就说数列是发散的.\\
常用的语言是:\[\lim_{n\rightarrow\infty}{x_n=a}\Leftrightarrow \forall_\varepsilon > 0,\exists N\in \bm{N}_+,\textit{当}n>N\textit{时},\textit{恒有}|x_n - a|<\varepsilon. \]
\end{theorem1}

\addtocounter{theoremcounter}{1}
\newtheorem{theorem2}{数列收敛}[theoremcounter]
\begin{theorem2}
若数列$\{a_n\}$收敛,则其任何子列${a_n}$也收敛,且$\lim\limits_{k\to\infty}{a_{n_k}} = \lim\limits_{n\rightarrow\infty}{a_n}.$ %NOTE:在内联模式中使用Lim的方法
\end{theorem2}

\addtocounter{theoremcounter}{1}
\newtheorem{theorem3}{收敛数列性质(唯一性)}[theoremcounter]
\begin{theorem3}
给出数列$\{x_n\}$,若$\lim\limits_{n\to\infty}{x_n}=a$存在,则$a$是唯一的.
\end{theorem3}

\addtocounter{theoremcounter}{1}
\newtheorem{theorem4}{收敛数列性质(有界性)}[theoremcounter]
\begin{theorem4}
若数列$\{x_n\}$,若$\lim\limits_{n\to\infty}{x_n}=a$存在,则$\{x_n\}$有界.
\end{theorem4}

\addtocounter{theoremcounter}{1}
\newtheorem{theorem5}{收敛数列性质(保号性)}[theoremcounter]
\begin{theorem5}
设数列$\{x_n\}$存在极限$a$,且$a>0($或$a<0)$,则存在正整数$N$,当$n>N$时,有$a_n>0($或$a_n<0)$.
\end{theorem5}

\newtheorem{theorem6}{推论}
\begin{theorem6}
设$a_n\geq 0(n=1,2,...)$,且$\lim\limits_{n\to\infty}{a_n}=a$,则$a\geq 0$.
\end{theorem6}

\setcounter{theoremcounter}{1}
\newtheorem{theorem7}{数列极限存在准则}[theoremcounter]
\begin{theorem7}
如果数列$\{x_n\},\{y_n\}$及$\{z_n\}$满足下列条件:\\
$(1)x_n\leq z_n \leq y_n(n=1,2,3,...);(2)\lim\limits_{n\to\infty}{x_n}=a,\lim\limits_{n\to\infty}{y_n}=a$,则数列$\{z_n\}$的极限存在,且$\lim\limits_{n\to\infty}{z_n}=a$.
\end{theorem7}

\addtocounter{theoremcounter}{1}
\newtheorem{theorem8}{数列极限存在准则}[theoremcounter]
\begin{theorem8}
单调有界数列必有极限.
\end{theorem8}

\subsection{函数极限定理}
\setcounter{theoremcounter}{1}
\subsubsection{函数极限定义}
设函数$f(x)$在点$x_0$的某一点去心邻域内有定义.若存在常数$A$,对于任意给定的$\varepsilon>0$(不论它多么小),总存在正整数$\delta$,使得当$0<|x-x_0|<\delta$时,对应的函数值$f(x)$都满足不等式$|f(x)-A|<\varepsilon$,则A就叫做函数$f(x)$,当$x\to x_0$时的极限,记为
\[ \lim\limits_{x\to x_0}{f(x)}=A\textit{或}f(x)\to A(x\to x_0). \]
写成$\varepsilon--\delta$语言是:$\lim\limits_{x\to x_0}{f(x)}=A\Leftrightarrow\forall_\varepsilon>0,\exists\delta>0,\textit{当}0<|x-x_0|<\delta\textit{时},|f(x)-A|<\varepsilon$.
\subsubsection{函数极限存在充要条件}
\[\lim\limits_{x\to x_0}{f(x)}=A \Leftrightarrow\lim\limits_{x\to x^{-}_0}{f(x)}=A,\lim\limits_{x\to x^{+}_0}{f(x)}=A, \]
\[\lim\limits_{x\to x_0}{f(x)}=A \Leftrightarrow f(x)=A+\alpha(x),\lim\limits_{x\to x_0}{\alpha(x)}=0.\]
\subsubsection{函数极限性质}
\paragraph{唯一性}如果极限$\lim\limits_{x\to x_0}{f(x)}$存在,那么极限唯一.
\paragraph{局部有界性} 略
\paragraph{局部保号性} 略
\subsubsection{无穷大与无穷小}
\paragraph{无穷大与无穷小比阶}设在同一自变量的变化过程中,
$\lim{\alpha(x)}=0,\lim{\beta(x)}=0,\textit{且}\beta(x)\neq 0,$则\\
\\
(1)若$\lim{\frac{\alpha(x)}{\beta(x)}}=0$,则称$\alpha(x)$是比$\beta(x)$\bm{高阶的无穷小},记为$\alpha(x)=o(\beta(x))$;\\
\\
(2)若$\lim{\frac{\alpha(x)}{\beta(x)}}=c\neq 0$,则称$\alpha(x)$是与$\beta(x)$\bm{同阶的无穷小};\\
\\
(3)若$\lim{\frac{\alpha(x)}{\beta(x)}}=1$,则称$\alpha(x)$是与$\beta(x)$\bm{等价的无穷小},记为$\alpha(x)\sim o(\beta(x))$;\\
\\
(4)若$\lim{\frac{\alpha(x)}{[\beta(x)]^k}}=c\neq 0$,则称$\alpha(x)$是$\beta(x)$的$k$阶无穷小;%12.26.2018 0:02 ERROR:" TeX capacity exceeded, sorry [main memory size=5000000] "
% 当去掉\bm{k阶无穷小}内容后可正常编译出pdf文件,猜测时由于过多使用\bm{}导致.
\subsubsection{无穷小运算规则}
\noindent (1)有限个无穷小的和是无穷小.\\
(2)有限函数与无穷小的乘积是无穷小.\\
(3)\underline{有限个}无穷小的乘积是无穷小.
\subsubsection{无穷小的计算}
\noindent\textcircled{1} $o(x^m)+o(x^n)=o(x^l),l=\min\{m,n\}$(加减法时低阶吸收``高阶'');\\
\textcircled{2} 乘法时阶数``累加'';\\
\textcircled{3} 非零常数不影响阶数;
\subsubsection{函数极限存在准则--夹逼准则}
如果函数$f(x),g(x)$及$h(x)$满足下列条件:
$(1)g(x)\leq f(x)\leq h(x);(2)\lim{g(x)}=A,\lim{h(x)}=A,\textit{则}\lim{f(x)}\textit{存在},\textit{且}\lim{f(x)}=A.$
\subsubsection{洛必达法则}
\paragraph{法则一}设(1)当$x\to a$(或$x\to \infty$)时,函数$f(x)$及$F(x)$都趋于零;\\
(2)$f^{\prime}(x)$及$F^{\prime}(x)$在点$a$的某去心邻域内(或者当$|x|>X$,此时$X$为充分大的正数)存在,且$F^{\prime}(x)\neq 0$;\\
(3)$\lim\limits_{x\to a}{\frac{f^{\prime}(x)}{F^{\prime}(x)}}(\textit{或}\lim\limits_{x\to \infty}{\frac{f^{\prime}(x)}{F^{\prime}(x)}})$存在或为无穷大,则$\lim\limits_{x\to a}{\frac{f(x)}{F(x)}}=\lim\limits_{x\to a}{\frac{f^{\prime}(x)}{F^{\prime}(x)}}(\textit{或}\lim\limits_{x\to \infty}{\frac{f(x)}{F(x)}}=\lim\limits_{x\to \infty}{\frac{f^{\prime}(x)}{F^{\prime}(x)}}).$
\paragraph{法则二}设(1)当$x\to a$(或$x\to \infty$)时,函数$f(x)$及$F(x)$都趋于无穷大;\\
(2)$f^{\prime}(x)$及$F^{\prime}(x)$在点$a$的某去心邻域内(或者当$|x|>X$,此时$X$为充分大的正数)存在,且$F^{\prime}(x)\neq 0$;\\
(3)$\lim\limits_{x\to a}{\frac{f^{\prime}(x)}{F^{\prime}(x)}}(\textit{或}\lim\limits_{x\to \infty}{\frac{f^{\prime}(x)}{F^{\prime}(x)}})$存在或为无穷大,则$\lim\limits_{x\to a}{\frac{f(x)}{F(x)}}=\lim\limits_{x\to a}{\frac{f^{\prime}(x)}{F^{\prime}(x)}}(\textit{或}\lim\limits_{x\to \infty}{\frac{f(x)}{F(x)}}=\lim\limits_{x\to \infty}{\frac{f^{\prime}(x)}{F^{\prime}(x)}}).$
\subsubsection{海涅定理(归结原理)}%12.26.2018 17:50.QUIZS:去心邻域的打印\mathring{U}
设$f(x)$在$\mathring{U}(x_0,\delta)$内有定义,则\\
\[\lim\limits_{x\to x_0}{f(x)}=A \textit{存在}\Leftrightarrow\textit{对任何以}x_0\textit{为极限的数列}\{x_n\}(x_n\neq x_0),\textit{极限}\lim\limits_{n\to \infty}{f(x_n)}=A\textit{存在.} \]

\subsection{函数连续与间断}
\subsubsection{连续定义}
两种定义方法:\\
(1)设$f(x)$在点$x_0$的某邻域内有定义,若\\
\[\lim\limits_{\Delta x\to 0}{\Delta y}=\lim\limits_{\Delta x\to 0}{[f(x_0+\Delta x)-f(x_0)]}=0,\]
则称函数在点$x_0$连续,点$x_0$成为$f(x)$的连续点.\\
(2)设函数f(x)在点$x_0$的某一邻域内有定义,且有$\lim\limits_{x\to x_0}{f(x)}=f(x_0)$,则称函数$f(x_0)$在点$x_0$处连续.(常用)
\subsubsection{间断点}
\begin{math}
\left\{
\begin{array}{ll}
\textit{可去间断点} & \\
\textit{跳跃间断点} &

\end{array}
\right\}
\end{math}
\end{document}
